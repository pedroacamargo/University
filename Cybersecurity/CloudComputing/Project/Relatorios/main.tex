\documentclass[11pt,a4paper]{report}
\usepackage[portuguese]{babel}
\usepackage[utf8]{inputenc} % define o encoding usado texto fonte (input)--usual "utf8" ou "latin1
\usepackage{graphicx} %permite incluir graficos, tabelas, figuras
\usepackage{subcaption}
\usepackage[title]{appendix}
\usepackage{listings}
\usepackage{color}
\usepackage{multicol}
\usepackage{indentfirst}
\usepackage{hyperref}
\usepackage{amsmath}
\usepackage{amssymb}
\usepackage{float}
\usepackage{enumerate}
\usepackage[shortlabels]{enumitem}
\usepackage[T1]{fontenc}
\usepackage{hyperref}



\definecolor{mygreen}{rgb}{0,0.6,0}
\definecolor{mygray}{rgb}{0.5,0.5,0.5}
\definecolor{mymauve}{rgb}{0.58,0,0.82}

\hypersetup{
    colorlinks=true,
    linkcolor=blue,
    urlcolor=black,
    }
    
    
    
\usepackage{bera}% optional: just to have a nice mono-spaced font

\definecolor{eclipseStrings}{RGB}{42,0.0,255}
\definecolor{eclipseKeywords}{RGB}{127,0,85}
    

\lstset{ %
  backgroundcolor=\color{white},   % choose the background color
  basicstyle=\footnotesize,        % size of fonts used for the code
  breaklines=true,                 % automatic line breaking only at whitespace
  captionpos=b,                    % sets the caption-position to bottom
  commentstyle=\color{mygreen},    % comment style
  escapeinside={\%*}{*)},          % if you want to add LaTeX within your code
  keywordstyle=\color{blue},       % keyword style
  stringstyle=\color{mymauve},     % string literal style
}





\title{Aplicações e Serviços de Computação em Nuvem Fase-1 \\
    
	\large Grupo nº33}

\author{Carlos Daniel Silva Fernandes \\ (PG59783) 
       \and Afonso Miguel da Silva Ribeiro \\ (PG60232)
       \and Pedro Augusto Ennes de Martino Camargo \\ (PG59791)
       \and Renato Pereira Garcia \\ (PG61542)
       \and Luís Filipe Pinheiro Silva \\ (PG59790)
}

\date{\today} % Data

\begin{document}

\raggedbottom % Evita espaçamentos forçados para preencher a página

\begin{minipage}{0.9\linewidth}
    \centering
    \includegraphics[width=0.4\textwidth]{uniMinho.png}\par
    \vspace{0.6 cm} % Reduz espaço vertical
    \href{https://www.uminho.pt/PT}
    {\scshape\LARGE Universidade do Minho} \par
    \vspace{1cm} % Reduz espaço vertical
    \href{https://www.eng.uminho.pt/pt/Estudar/_layouts/15/UMinho.PortaisUOEI.UI/Pages/CatalogoCursoDetail.aspx?itemId=5692&catId=16}
    {\scshape\Large Mestrado em Cibersegurança} \par
    \vspace{0.2cm} % Reduz espaço vertical
    {\scshape\Large Mestrado em Engenharia Informática} \par

    \vspace{-0.5cm} % Reduz espaço antes do título
    \maketitle
    \vspace{-1cm} % Reduz espaço depois do título

   
\end{minipage}

\vspace{-0.5cm} % Reduz espaço antes do índice
\renewcommand{\baselinestretch}{0.9}\normalsize % Compacta o espaçamento entre linhas

\tableofcontents

\pagebreak


	\chapter{Contextualização}

        \section{Visão Geral da Aplicação}

        Numa primeira fase deste projeto, é feito o estudo da aplicação de tracking de voos airtrail.
        Neste relatório estão presentes detalhes sobre a arquitetura da aplicação, as suas funcionalidades e APIs, bem como uma reflexão e discussão sobre potenciais pontos de falha da aplicação, gargalos de desempenho e possíveis dificuldades na instalação.
        
        O principal objetivo da aplicação \textbf{AirTrail} é fornecer um sistema completo de \textbf{tracking de viagens aéreas}, permitindo que o utilizador registe e acompanhe todas as viagens realizadas num determinado período de tempo.

        \subsection{Funcionalidades Principais}

        \begin{itemize}
        \item \textbf{Adição Manual de Viagens:}  
        O utilizador pode adicionar uma viagem manualmente através de formulários disponibilizados pela própria aplicação.

        \item \textbf{Importação de Dados:}  
        O \textbf{AirTrail} permite importar ficheiros provenientes de outras plataformas de gestão de voos, facilitando a migração e centralização de dados.  

        \item \textbf{Exportação de Dados:}  
        O utilizador pode exportar os seus registos de voo, permitindo criar cópias de segurança (\textit{backups}) ou transferir os dados para outras aplicações compatíveis.

        \item \textbf{Estatísticas de Voo:}  
        A aplicação gera automaticamente estatísticas e métricas baseadas nas viagens registadas, fornecendo ao utilizador uma visão geral do seu histórico de voos.
        \end{itemize}

        \section{Principais Tecnologias}
        \noindent \textbf{Base de Dados:} \textit{PostgreSQL}; 
        \textbf{Web Server:} \textit{SvelteKit} + \textit{TypeScript};
        \textbf{Backend:} Integrado no frontend via \textit{SvelteKit};
        \textbf{Acesso à BD:} \textit{Kysely};
        \textbf{Schema/Migrações:} \textit{Prisma}.

        \section{Instalação da Aplicação}

        A aplicação \textit{AirTrail} pode ser instalada de duas formas principais: manualmente ou através de \textit{containers} utilizando \textit{Docker}.

        A \textbf{instalação manual} é um processo mais trabalhoso e pouco fiável para sistemas de maior escala, uma vez que a sua replicação e manutenção tornam-se mais complexas e propensas a erros humanos. Além disso, apresenta menor portabilidade entre diferentes ambientes.

        Por outro lado, a \textbf{instalação containerizada} com \textit{Docker} oferece maior portabilidade, facilita o processo de implantação e reduz significativamente o risco de inconsistências. Esta abordagem também permite uma escalabilidade mais simples e eficiente, sendo a mais recomendada para ambientes de produção.


        \section{Arquitetura e Componentes}

        A arquitetura adotada é do tipo \textbf{cliente-servidor}, em que o cliente interage com o servidor web que, por sua vez, comunica com a base de dados.
        
        A infraestrutura é composta por dois containers Docker que comunicam entre si através de uma \textit{bridge network} criada automaticamente pelo \textit{docker compose}:

        O web server e a base de dados são implementados nos containers airtrail e airtrail\_db, respetivamente.


        % A comunicação entre a base de dados e o servidor web é facilitada através de um \textbf{ORM} (\textit{Object-Relational Mapping}), utilizando as bibliotecas \textit{Prisma} e \textit{Kysely} para abstrair e simplificar o acesso aos dados.

        \section{APIs fornecidas pela aplicação}

        \subsection{API REST}

         A \textbf{API REST} da aplicação \textbf{AirTrail} fornece um conjunto de endpoints para gestão das viagens registadas por cada utilizador.  
        Abaixo estão listadas as principais rotas:

        {\small
        \begin{verbatim}
        GET  /flight/list       -> Lista todos os voos do utilizador.
        GET  /flight/get/[id]   -> Detalhes de um voo específico.
        POST /flight/save       -> Cria ou atualiza um voo existente.
        POST /flight/delete     -> Remove um voo pelo seu ID.
        \end{verbatim}
        }

        \subsection{APIs Externas}

        A aplicação \textbf{AirTrail} faz uso de serviços externos para obter e enriquecer informações sobre voos e aeronaves.  

        \begin{itemize}
        \item \textbf{Adsbdb:}  
        Serviço utilizado para realizar \textit{fetch} de informações associadas a \textit{flight numbers}.  

        \item \textbf{AeroDataBox:}  
        Alternativa à Adsbdb, permitindo que o utilizador opte por uma integração premium.  
        \end{itemize}


    \chapter{Análise de Riscos}
        
        \section{Ponto Único de Falha}

        O principal risco na arquitetura da aplicação \textbf{AirTrail} é o \textbf{Web Server}.  
        Caso o servidor falhe por qualquer motivo, todos os clientes conectados perdem imediatamente o acesso à aplicação.  

        Além disso, a utilização de APIs externas, representa outro ponto de falha potencial.  
        Caso este serviço fique indisponível, o funcionamento do \textbf{AirTrail} poderá ser afetado.  

        Para mitigar o risco, recomenda-se a implementação de \textbf{redundância}, \textbf{replicação} e \textbf{failover}, garantindo alta disponibilidade do serviço.

        \section{Gargalos de Desempenho}

        À medida que o número de utilizadores cresce, o servidor pode passar por gargalos de desempenho, resultando em tempos de resposta mais lentos ou até falhas.

        Com apenas uma instância de servidor web, o sistema não possui escalabilidade horizontal. Assim, em situações de aumento de tráfego, o servidor pode atingir o limite da sua capacidade, afetando o desempenho global da aplicação.

        O comando \texttt{docker stats} foi utilizado para monitorizar o consumo de recursos dos containers em execução.

        \begin{figure}[H]
            \centering
            \includegraphics[width=1\textwidth]{dockerStatus.png}
            \caption{Consumo de recursos dos containers.}
            \label{fig:airtrail-containers}
        \end{figure}

        Comparando com as especificações mínimas recomendadas pela documentação oficial do projeto (\textbf{2~CPU cores}, \textbf{2~GB de RAM} e \textbf{10~GB de espaço em disco}), observa-se que o consumo real é significativamente inferior aos limites estabelecidos.  
Isso indica que o AirTrail é uma aplicação leve e eficiente para pequenos ambientes de produção ou testes locais.


        % \subsection{Dependências Externas}

        % A aplicação depende de serviços externos para o seu correto funcionamento.  
        % Um exemplo crítico é o uso da API \textbf{ADSBDB} para obtenção de informações de voos.  

        % Caso este serviço deixe de ser mantido, fique desatualizado ou indisponível, o funcionamento do \textbf{AirTrail} poderá ser severamente afetado.  
        % Recomenda-se a existência de \textbf{mecanismos de fallback} ou alternativas configuráveis (como o \textbf{AeroDataBox}) para minimizar o impacto de eventuais falhas externas.

    
	% \chapter{Conclusão}
    %     Durante o desenvolvimento desta fase, o principal desafio passou por compreender e implementar corretamente as curvas de \textbf{Bezier} e de \textbf{Catmull-Rom}, bem como ajustar as transformações de forma a garantir um movimento natural dos objetos.\\
    %     Apesar destas adversidades, consideramos que concluímos o trabalho com a qualidade exigida pela equipa docente.

\end{document}