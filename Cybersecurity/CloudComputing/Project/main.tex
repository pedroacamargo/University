\documentclass[11pt,a4paper]{report}
\usepackage[portuguese]{babel}
\usepackage[utf8]{inputenc} % define o encoding usado texto fonte (input)--usual "utf8" ou "latin1
\usepackage{graphicx} %permite incluir graficos, tabelas, figuras
\usepackage{subcaption}
\usepackage[title]{appendix}
\usepackage{listings}
\usepackage{color}
\usepackage{multicol}
\usepackage{indentfirst}
\usepackage{hyperref}
\usepackage{amsmath}
\usepackage{amssymb}
\usepackage{float}
\usepackage{enumerate}
\usepackage[shortlabels]{enumitem}
\usepackage[T1]{fontenc}
\usepackage{hyperref}



\definecolor{mygreen}{rgb}{0,0.6,0}
\definecolor{mygray}{rgb}{0.5,0.5,0.5}
\definecolor{mymauve}{rgb}{0.58,0,0.82}

\hypersetup{
    colorlinks=true,
    linkcolor=blue,
    urlcolor=black,
    }
    
    
    
\usepackage{bera}% optional: just to have a nice mono-spaced font

\definecolor{eclipseStrings}{RGB}{42,0.0,255}
\definecolor{eclipseKeywords}{RGB}{127,0,85}
    

\lstset{ %
  backgroundcolor=\color{white},   % choose the background color
  basicstyle=\footnotesize,        % size of fonts used for the code
  breaklines=true,                 % automatic line breaking only at whitespace
  captionpos=b,                    % sets the caption-position to bottom
  commentstyle=\color{mygreen},    % comment style
  escapeinside={\%*}{*)},          % if you want to add LaTeX within your code
  keywordstyle=\color{blue},       % keyword style
  stringstyle=\color{mymauve},     % string literal style
}





\title{Aplicações e Serviços de Computação em Nuvem Fase-1 \\
    
	\large Grupo nº33}

\author{Carlos Daniel Silva Fernandes \\ (A102499) 
       \and Afonso Miguel da Silva Ribeiro \\ (PG60232)
       \and Pedro Augusto Ennes Camargo \\ (A102504)
       \and Renato Pereira Garcia \\ (A101987)
}

\date{\today} % Data

\begin{document}

\raggedbottom % Evita espaçamentos forçados para preencher a página

\begin{minipage}{0.9\linewidth}
    \centering
    \includegraphics[width=0.4\textwidth]{uniMinho.png}\par
    \href{https://www.uminho.pt/PT}
    {\scshape\LARGE Universidade do Minho} \par
    \vspace{0.2cm} % Reduz espaço vertical
    \href{https://www.eng.uminho.pt/pt/Estudar/_layouts/15/UMinho.PortaisUOEI.UI/Pages/CatalogoCursoDetail.aspx?itemId=5692&catId=16}
    {\scshape\Large Mestrado em Cibersegurança} \par

    \vspace{-0.5cm} % Reduz espaço antes do título
    \maketitle
    \vspace{-1cm} % Reduz espaço depois do título

   
\end{minipage}

\vspace{-0.5cm} % Reduz espaço antes do índice
\renewcommand{\baselinestretch}{0.9}\normalsize % Compacta o espaçamento entre linhas

\tableofcontents

\pagebreak


	\chapter{Contextualização}

        \section{Visão Geral da Aplicação}

        Numa primeira fase deste projeto, é feito o estudo da aplicação de tracking de voos airtrail.
        Neste relatório estão presentes detalhes sobre a arquitetura da aplicação, as suas funcionalidades e APIs, bem como uma reflexão e discussão sobre potenciais pontos de falha da aplicação, gargalos de desempenho e possíveis dificuldades na instalação.

        \section{Principais Tecnologias}

        \begin{itemize}
        \item \textbf{Base de Dados:} \textit{PostgreSQL}
        \item \textbf{Web Server:} Desenvolvido com \textit{SvelteKit} e \textit{TypeScript}
        \item \textbf{Backend:} Integrado no próprio frontend através do \textit{SvelteKit}
        \item \textbf{Acesso à Base de Dados:} \textit{Kysely} (SQL Builder)
        \item \textbf{Schema e Migrações:} Geridas com \textit{Prisma}
        \end{itemize}

        \section{Arquitetura}

        A arquitetura adotada é do tipo \textbf{cliente-servidor}, em que o cliente interage com o servidor web que, por sua vez, comunica com a base de dados.

        \section{Componentes}

        \begin{itemize}
        \item \textbf{Web Server:} Container \texttt{airtrail}
        % \begin{itemize}
            % \item \textit{Função:} Expor a API REST que permite aos utilizadores interagir com a base de dados.
        % \end{itemize}

        \item \textbf{Base de Dados:} Container \texttt{airtrail\_db}
        % \begin{itemize}
        %     \item \textit{Função:} Armazenar os dados relacionais da aplicação AirTrail.
        % \end{itemize}
        \end{itemize}

        \section{Infraestrutura}

        A infraestrutura é composta por dois containers Docker que comunicam entre si através de uma \textit{bridge network} criada automaticamente pelo \textit{docker compose}:

        \begin{verbatim}
        NETWORK ID     NAME                        DRIVER    SCOPE
        b16b80e06117   airtrail_default            bridge    local
        \end{verbatim}

        A comunicação entre a base de dados e o servidor web é facilitada através de um \textbf{ORM} (\textit{Object-Relational Mapping}), utilizando as bibliotecas \textit{Prisma} e \textit{Kysely} para abstrair e simplificar o acesso aos dados.

    \chapter{Arquitetura e principais componentes}
        A aplicação tem uma arquitetura clinte-servidor, com o servidor ligado a uma base de dados PostgresSQL. Tanto o servidor como a base de dados correm em containers Docker, e comunicam através de uma bridge network criada automaticamente pelo docker compose. Entre a base de dados e o web server é utilizado um ORM (Object-Relational Mapping) para facilitar a comunicação entre os dois componentes

        \section{Web Server}

        \section{Base de dados}

	\chapter{Conclusão}
        Durante o desenvolvimento desta fase, o principal desafio passou por compreender e implementar corretamente as curvas de \textbf{Bezier} e de \textbf{Catmull-Rom}, bem como ajustar as transformações de forma a garantir um movimento natural dos objetos.\\
        Apesar destas adversidades, consideramos que concluímos o trabalho com a qualidade exigida pela equipa docente.

\end{document}